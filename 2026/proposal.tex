\documentclass[10pt]{article}
\usepackage[colorlinks=true,urlcolor=blue,citecolor=blue,pdfstartview=FitH]{hyperref}
\usepackage[a4paper]{geometry}

\begin{document}

\title{ICAPS 2026 Workshop on Heuristics and Search for Domain-independent
Planning (HSDIP)}

\date{}
\author{}

\maketitle

We propose a continuation of the ICAPS workshop series ``Heuristics
and Search for Domain-independent Planning'' at ICAPS 2026.

\section*{Description}
Heuristic search is one of the main approaches in domain-independent
planning.  We look for contributions that help us better
understand the ideas underlying current heuristics, their limitations,
and ways for overcoming them.
Since the search algorithm plays an equally important role in the
approach, we also invite contributions on new ideas on search
techniques for domain-independent planning, as well as the synergy
between heuristics and search.

Contributions do not have to show that a new approach ``beats the competition''.
While performance measured in the number of evaluated nodes, time, and solution
quality remains relevant, in this workshop we seek above all crisp and
meaningful ideas and understanding.
We are interested in all variations of domain-independent planning
such as classical planning, temporal planning, hybrid planning, planning under
uncertainty, adversarial planning, or (model-based) reinforcement learning.

The HSDIP workshop has always been welcoming of multidisciplinary work,
for example, drawing inspiration from operations research (like row and
column generation algorithms), convex optimization (like gradient
optimization for hybrid planning), constraint programming or
satisfiability, or applications of machine learning in heuristic search
(e.g., learning heuristics, or heuristic selection). We will keep this
stance and promote it in the Call for Participation.

\section*{Relevance to ICAPS}

The workshop on heuristics and search for domain-independent planning (HSDIP) is
the successor of the workshop on heuristics for domain-independent planning
(HDIP), a biennial event that took place in conjunction with ICAPS during the
years 2007, 2009, and 2011. Starting with the HSDIP workshop in 2012, the
workshop series became an annual event.
Many ideas presented at HDIP and HSDIP workshops have led to contributions at major
conferences and pushed the frontier of research on heuristic planning in several
directions, both theoretically and practically. Throughout the years, HSDIP has
consistently had a large number of high-quality submissions and a large workshop
participation, including the virtual workshops from 2020 to 2022, which justifies its
continuation as an annual event.

\section*{Format}

The workshop is planned to have a full 1-day format. We aim to allow
all workshop contributors to present orally as well as in a poster
session, if the conference can accommodate posters for the workshops.

Many workshops these days have become a lot like mini-conferences in
the sense that they mostly consist of traditional presentations of
papers that could also be given at the main conference and ended up at
the workshop due to deadline considerations, rejection from the main
program, or as part of a ``papers from other conferences'' scheme.
While we are definitely still happy to see such presentations,
following a discussion at HSDIP 2025 we want to make a concerted
effort this year to make the workshop more ``workshop-like'' with a
lot of breathing room for half-baked ideas, failed efforts and open
challenges, and with longer discussions for the talks, as well as
panels or other interactive components.

To make this general plan more concrete, we plan to solicit additional
ideas in this direction from the community and conduct a poll to gauge
the interest in different suggestions. We will also put special emphasis
on exploratory aspects of the workshop and interactivity in the call
for papers.

\section*{Diversity}

The HSDIP organizing committee is pursuing several strategies to promote
diversity in the HSDIP community. So far, these include,

\begin{itemize}
\item Inviting a diverse range of potential organizers for the workshop,
  both in terms of under-represented groups as well as academic progression.
\item Committing to a diverse panel of speakers or invited talks (depending
  on the final chosen format).
\item Actively promoting the workshop to interested undergraduate
  students specializing in AI, to better promote the field to a wider audience.
\end{itemize}

This list is by no means complete, and we will continue to innovate on what
we can do as an individual workshop to promote diversity both within the scope
of HSDIP, and ICAPS more broadly.

\section*{Organization}

We plan to organize the workshop in a team of four people.
For reviewing, we will take a major part of the involved effort.
If needed to handle the workload, we will invite a small number of additional
external reviewers.
These would be carefully briefed regarding the goals of the workshop,
in particular being open to submissions that don't look like typical
conference papers.
We expect to have around 10--20 submissions, which can easily be handled in that
way, particularly considering that reviewing for workshops is generally
lightweight.

\newcommand{\organizer}[4]{\href{#2}{#1} (\href{mailto:#3}{#3})\\{#4}}
\begin{itemize}

%% Organizers sorted by last name.

\item \organizer{Malte
  Helmert}{https://ai.dmi.unibas.ch/people/helmert/}{malte.helmert@unibas.ch}
Malte Helmert is a professor at the University of Basel, Switzerland.
His main research focus is in heuristics and heuristic search for
classical planning, with a heavy emphasis on theory alongside
algorithms. In recent years, his main focus was on certifying planning
algorithms, lifted and generalized planning, connecting heuristic
search to mathematical programming (LPs/MIPs), and unsolvability.

\item \organizer{Arnaud Lequen}{https://mrlab.ai/arnaud-lequen/}{arnaud.lequen@liu.se}
Arnaud Lequen is a postdoctoral researcher at Linköping University, Sweden,
where he works on symbolic methods for automated planning. He obtained a PhD
at IRIT in Toulouse, France, in November 2024, under the supervision of Prof.\
Martin C.\ Cooper. His thesis focused on theoretical and practical
aspects of extracting knowledge from planning models, prior to the solving phase.

\item \organizer{Alison Paredes}{https://mulab.ai/member/alison.paredes/}{20asp3@queensu.ca}
Alison Paredes is a PhD candidate at Queen's University, working with
Prof.\ Christian Muise, and Pathways Intern at NASA Ames.
Her research is on planning as a source of sampling bias, with applications
wherever planner-generated data is used as input to statistical learning
models, including LLM reasoning, multi-robot path finding, and learned
heuristics.
She earned her master's degree in Computer Science at the University of New
Hampshire.

\item \organizer{Devin Thomas}{https://dwthomas.github.io/}{dwt@cs.unh.edu}
Devin Thomas is a PhD student at the University of New Hampshire.
His primary research focus is on heuristic search algorithms for planning under
time pressure.
\end{itemize}

\section*{Estimated Interest}

Previous years have shown an average number of about $12$ workshop
speakers and an audience of approximately $40$ attendees. The
expectation is for a similar amount this year.

\end{document}
