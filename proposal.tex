\documentclass[10pt]{article}

\newcommand{\commentout}[1]{}
\usepackage[colorlinks=true,urlcolor=blue,citecolor=blue,pdfstartview=FitH]{hyperref}

\usepackage{geometry}
\usepackage{layout}
\usepackage{titling}

\setlength{\droptitle}{-5em}   % This is your set screw

\begin{document}

\title{ICAPS 2021 Workshop on Heuristics and Search for Domain-independent
Planning (HSDIP)%\\ \vspace*{0.7cm} Workshop Proposal
}
\date{}

\author{}

\maketitle

\vspace{-2cm}


We propose a continuation of the ICAPS workshop series ``Heuristics
and Search for Domain-independent Planning'' at ICAPS 2021.

\section*{Workshop Description}
Heuristic search is one of the main approaches in domain-independent
planning.  We look for contributions that would help us better
understand the ideas underlying current heuristics, their limitations,
and ways for overcoming them.
%
Since the search algorithm plays an equally important role in the
approach, we also invite contributions on new ideas on search
techniques for domain-independent planning, as well as the synergy
between heuristics and search.

Contributions do not have to show that a new approach ``beats the competition''.
While performance measured in the number of evaluated nodes, time, and solution
quality remains relevant, in this workshop we seek above all crisp and
meaningful ideas and understanding.
%
We are interested in all variations of domain-independent planning
such as classical planning, temporal planning, hybrid planning, planning under
uncertainty, adversarial planning or (model-based) reinforcement learning.

The HSDIP workshop has always been welcoming of multidisciplinary work,
for example, drawing inspiration from operations research (like row and
column generation algorithms), convex optimization (like gradient
optimization for hybrid planning), constraint programming or
satisfiability, or applications of machine learning in heuristic search
(e.g., learning heuristics, or heuristic selection). We will keep this
stance and promote it in the Call for Participation, particularly as ICAPS 2021
will continue the special track on planning \& learning.

\section*{Diversity}
The HSDIP workshop has always been an inclusive environment, yet an analysis
of the diversity among participants of previous years shows that a large portion
of authors is male and either European or North American. Although we had
submissions from other countries, such as South America, Asia, Australia, or
by women in general, they form the minority.

However, we feel that promoting diversity just by explicitly mentioning that we
put an emphasis on underrepresented groups is the wrong approach. Instead, we
want to emphasize that a career choice as a computer scientist is a strong choice,
regardless of gender and nationality. For that we plan to have an invited talk
given by a speaker who represents that diversity and successful careers can go
hand in hand. Sylvie Thi{\'e}baux expressed her interest. Sylvie is a professor
for Computer Science at the Australian National University, former ICAPS
president, and Councilor of AAAI.
{\color{red} Daniel: should we do this again? Gabi comes into mind :)}

\section*{Relevance to ICAPS 2021}

The workshop on heuristics and search for domain-independent planning (HSDIP) is
the successor of the workshop on heuristics for domain-independent planning
(HDIP), a biennial event that took place in conjunction with ICAPS during the
years 2007, 2009, and 2011. Starting with the HSDIP workshop in 2012, the
workshop series became an annual event.
Many ideas presented at HDIP and HSDIP workshops have led to contributions at
major conferences and pushed the frontier of research on heuristic planning in
several directions, both theoretically and practically.
Throughout the years, HSDIP has consistently had a large number of
high-quality submissions and a large workshop participation, including 2020,
which justifies its continuation as an annual event.

\section*{Workshop Format}

The workshop is planned to have a full 1-day format, but the precise
format is to be decided as a function of the contributions received.
We aim to allow all workshop contributors to present orally, and will
adapt the presentation timing to suit. In addition, we plan to include
1-2 dedicated discussion sessions where the audience members are
encouraged to participate. All these decisions obviously depend on how ICAPS 2021
is actually held, virtual, hybrid, or in-person as currently planned. We
will adapt the format accordingly.

Following the success of the workshop for the past years and the fact
that the majority of workshop submissions are accepted, we would like to reduce
the reviewing load of the community by not having a program
committee. Acceptance decisions will be made directly by the
organizers. We expect to have around 10-20 submissions, which can
easily be handled by us, particularly considering that reviewing for
workshops is generally light. Between us, we have sufficient
competence in the topic of the workshop to make informed decisions,
but should the need arise, we can call on additional reviewing
expertise.

\section*{Organizers}

\begin{itemize}

\item \href{https://ai.dmi.unibas.ch/people/eriksson/}{Salom\'{e} Eriksson}
(\href{mailto:salome.eriksson@unibas.ch}{salome.eriksson@unibas.ch})\\
Salom\'{e} Eriksson is a postdoctoral researcher at the University of 
Basel, where she also received her PhD in Computer Science in 2019. Her
current research as well as her dissertation focuses on how we can
create certificates for unsolvable planning tasks in order to increase
trustworthiness of state-of-the-art planning systems.

\item \href{https://ai.dmi.unibas.ch/people/ferber/}{Patrick Ferber}
(\href{mailto:patrick.ferber@unibas.ch}{patrick.ferber@unibas.ch})\\
Patrick Ferber is a PhD student in the AI group at the University of 
Basel and in the Foundations of Artificial Intelligence group at 
Saarland University. His research interest is on machine learning for 
planning with a focus on learning heuristic functions.

\item \href{http://cs.fel.cvut.cz/en/people/fiserdan}{Daniel Fi\v{s}er}
(\href{mailto:danfis@danfis.cz}{danfis@danfis.cz})\\
Daniel Fi\v{s}er is a PhD student at Czech Technical University in Prague.
His main research interest is classical planning, in particular inference and
application of state invariants.

\item \href{http://fai.cs.uni-saarland.de/gnad/}{Daniel Gnad}
(\href{mailto:gnad@cs.uni-saarland.de}{gnad@cs.uni-saarland.de})\\
Daniel Gnad is a PhD student at the Foundations of Artificial
Intelligence group at Saarland University. His research is in the field
of classical AI planning, mainly focusing on the newly developed concept
of star-topology decoupled search. Besides, he has also worked on partial
delete-relaxation.

\item \href{http://ai.cs.unibas.ch/people/pommeren}{Florian Pommerening}
   (\href{mailto:florian.pommerening@unibas.ch}{florian.pommerening@unibas.ch})\\
Florian Pommerening is a postdoctoral researcher in the AI group at the
University of Basel, Switzerland, where he completed his PhD from 2012
to 2017. His main research interest is classical automated planning. For
his PhD thesis, he used linear and mixed integer programs to
automatically derive heuristic functions.

\item \href{http://www.informatik.uni-freiburg.de/~speckd/}{David Speck}
(\href{mailto:speckd@informatik.uni-freiburg.de}{speckd@informatik.uni-freiburg.de})\\
David Speck is a PhD student in the Foundations of Artificial Intelligence
group at the University of Freiburg, Germany. His current research focuses on
the use of symbolic data structures for automated planning.

\item \href{https://fai.cs.uni-saarland.de/torralba/}{{\'A}lvaro Torralba}
  (\href{mailto:torralba@cs.uni-saarland.de}{torralba@cs.uni-saarland.de})\\ \'Alvaro
  Torralba is a postdoctoral researcher in the FAI group of Saarland University. He
  received his PhD from Universidad Carlos III de Madrid in 2015 for his dissertation on
  symbolic search and abstraction heuristics for classical planning. His main research
  interests are on heuristic search and automated planning.


\end{itemize}

\section*{Estimated Interest}
Previous years have shown an average number of about $12$ workshop
speakers and an audience of approximately $40$ attendees. The
expectation is for a similar amount this year.


\end{document}
