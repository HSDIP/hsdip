\documentclass[10pt]{article}

\newcommand{\commentout}[1]{}
\usepackage[colorlinks=true,urlcolor=blue,citecolor=blue,pdfstartview=FitH]{hyperref}


\begin{document}

\title{ICAPS 2019 Workshop on \\ Heuristics and Search for Domain-independent
Planning (HSDIP)\\ \vspace*{0.7cm} Workshop Proposal
}
\date{}

\author{}

\maketitle


We propose a continuation of the ICAPS workshop series ``Heuristics
(and Search) for Domain-independent Planning'' at ICAPS 2019.

\section*{Workshop Description}
Heuristic search is one of the main approaches in domain-independent
planning.  We look for contributions that would help us better
understand the ideas underlying current heuristics, their limitations,
and ways for overcoming them.
%
Since the search algorithm plays an equally important role in the
approach, we also invite contributions on new ideas on search
techniques for domain-independent planning, as well as the synergy
between heuristics and search.

Contributions do not have to show that a new approach ``beats the competition''.
While performance measured in the number of evaluated nodes, time, and solution
quality remains relevant, in this workshop we seek above all crisp and
meaningful ideas and understanding.
%
We are interested in all variations of domain-independent planning
such as classical planning, temporal planning, hybrid planning, planning under
uncertainty, adversarial planning or (model-based) reinforcement learning.

As ICAPS 2019 will host a new special track on planning \& learning we are interested
too in receiving submissions with relation to applications of machine learning to
obtain heuristics for MDPs or other settings relevant to reinforcement learning\footnote{Other
ideas here?}. Keeping in line with past editions, we are highly interested in multidisciplinary work that
draws inspiration from operations research (like row and column generation algorithms),
convex optimization (like gradient optimization for hybrid planning),
constraint programming or satisfiability, introducing to the planning community
new potentially interesting techniques.

%\newpage

\section*{Relevance to ICAPS 2019}

The workshop on heuristics and search for domain-independent planning
(HSDIP) is the successor of the workshop on heuristics for
domain-independent planning (HDIP), a biennial event that took place
in conjunction with ICAPS during the years 2007, 2009, and 2011. The
HDIP workshop was a very successful event every time. Many ideas
presented at HDIP workshops have led to contributions at major
conferences and pushed the frontier of research on heuristic planning
in several directions, both theoretically and practically.
%
With the HSDIP workshops in 2011, 2012, 2013, 2014, 2015, 2016, 2017 and 2018 the
workshop series became an annual event. The fact that last year's HSDIP again had very
high quality contributions and that it was one of the \textbf{largest}
\footnote{Was this the case in 2018?}
workshops held shows a sufficient amount of interest in the ICAPS community, justifying its
continuation as annual event.

\section*{Workshop Format}

The workshop is planned to have a full 1-day format, but the precise
format is to be decided as a function of the contributions received.
We strive to have two types of presentations: long and short (30 and
15 minutes, including discussion), as well as 1-2 dedicated discussion
sessions where the audience members are encouraged to participate.

Following the success of the workshop for the last 5 years and the fact
that the majority of workshop submissions are accepted, we would like to reduce
the reviewing load of the community by not having a program
committee. Acceptance decisions will be made directly by the
organizers. We expect to have around 10-20 submissions, which can
easily be handled by us, particularly considering that reviewing for
workshops is generally light. Between us, we have sufficient
competence in the topic of the workshop to make informed decisions,
but should the need arise, we can call on additional reviewing
expertise.

\section*{Organizers}

\begin{itemize}

\item \href{http://http://users.cecs.anu.edu.au/~patrik/}{Patrik Haslum}
(\href{mailto:patrik.haslum@anu.edu.au}{patrik.haslum@anu.edu.au})
Still works on classical Planning. Kind of.

\item \href{http://}{Daniel Gnad}
(\href{mailto:gnad@cs.uni-saarland.de}{gnad@cs.uni-saarland.de})\\
Daniel Gnad is a Ph.D.\ student at the Foundations of Artificial
Intelligence chair at Saarland University, hosted by J\"org
Hoffmann. He obtained his Bachelors and Masters degrees from the same
University. His research is in the field of classical AI planning,
mainly focusing on the newly developed concept of star-topology
decoupled search. Besides of that, he has also worked on partial
delete-relaxation.

\item \href{http://findanexpert.unimelb.edu.au/display/person778610#tab-overview}{Miquel Ram\'{i}rez}
(\href{mailto:miguel.ramirez@unimelb.edu.au}{miguel.ramirez@unimelb.edu.au})\\
Miquel is a Research Fellow at the University of Melbourne. Previously, he was
a Research Fellow at the Australian National University and the Royal Melbourne
Institute of Technology. He
received his PhD in Computer Science in 2012 for a dissertation
bringing together Automated Planning and Plan Recognition. His current
research is on planning for numeric domains with dynamical constraints, hybrid control
systems and supervisory control for general simulation frameworks.

 \item \href{http://ai.cs.unibas.ch/people/pommeren}{Florian Pommerening}
   (\href{mailto:florian.pommerening@unibas.ch}{florian.pommerening@unibas.ch})\\
   Florian is a PhD student in the AI group at the University of Basel,
   Switzerland, where he started in 2012 after receiving his MSc in
   Computer Science from the University of Freiburg, Germany. His main
   research interest is classical automated planning. For his PhD
   thesis, he is using linear and mixed integer programs to
   automatically derive heuristic functions.

 \item \href{http://ai.cs.unibas.ch/people/seipp}{Jendrik Seipp}
   (\href{mailto:jendrik.seipp@unibas.ch}{jendrik.seipp@unibas.ch})\\
   Jendrik is a postdoctoral researcher in the Artificial Intelligence
   group at the University of Basel, Switzerland. His PhD thesis focuses
   on Cartesian abstraction heuristics and saturated cost partitioning
   for optimal classical planning. In addition to these topics, his main
   research interests are planner portfolios and potential heuristics.

% \item \href{https://ti.arc.nasa.gov/profile/j-benton/}{J. Benton}
%   (\href{mailto:j.benton@nasa.gov}{j.benton@nasa.gov})\\
%   J. Benton is a senior research scientist in the Planning and Scheduling Group at the NASA Ames Research Center. His research interests include planning with incomplete domain knowledge, over-subscription planning, temporal planning, probabilistic planning, and planning applications. J. co-organized the ICAPS Workshop on Heuristics and Search for Domain-independent Planning (HSDIP) from 2013 through 2016, and is the Tutorial co-chair for ICAPS 2017.


%\item \href{http://www.sift.net/staff/dan-bryce}{Daniel Bryce}
%  (\href{mailto:dbryce@sift.net}{dbryce@sift.net})\\
%Daniel is a senior researcher at Smart Information Flow Technologies (SIFT, LLC). His
%  primary research interest is non-classical planning, including
%  planning under uncertaininty and hybrid planning.  He received his PhD for a
%  dissertation on automated planning from Arizona State University in
%  2007. He was a co-organizer of workshops in
%  ICAPS 2008 and 2010.
%



%\item \href{https://resedit.watson.ibm.com/researcher/view.php?person=ibm-Michael.Katz1}{Michael Katz}
% (\href{mailto:michael.katz1@ibm.com}{michael.katz1@ibm.com})\\
% Michael is a researcher at IBM T.J. Watson Research Center, NY, USA. His
% primary research interest is in heuristic search for domain independent planning.
% He received his PhD for a dissertation on heuristics for domain independent
% planning from Technion in 2010.
% He was a co-organizer of the 2011, 2013, 2014, 2015, and 2016 H(S)DIP
% workshops.

%\item \href{http://people.eng.unimelb.edu.au/nlipovetzky/}{Nir Lipovetzky}
%  (\href{mailto:nir.lipovetzky@unimelb.edu.au}{nir.lipovetzky@unimelb.edu.au})\\
%Nir is a Lecturer at The University of Melbourne.
%His research is in the area of Automated Planning, with a special focus on width-based algorithms
%to introduce different approaches to the problem of inference in planning. He was co-publicity chair of ICAPS 2010, co-organizer
%of the HSDIP workshop at ICAPS 2015, 2016, 2017, co-organizer of the first unsolvability IPC 2016 track, and co-chair of ICAPS 2017 journal track.

%\item \href{http://ai.cs.unibas.ch/people/}{Guillem Franc\`{e}s}
%     (\href{mailto:guillem.frances@unibas.ch}{guillem.frances@unibas.ch})\\
%Guillem is a Postdoctoral Researcher at the Artificial Intelligence group at the University of Basel, Switzerland.
%He received his PhD from Universitat Pompeu Fabra, Barcelona, in 2017, under the supervision of H\'{e}ctor Geffner,
%for a dissertation on effective techniques to tackle expressive modeling languages for classical planning.
%His current research interests are focused on simulation-based planning techniques and on
%the impact of modeling choices on the efficiency of planning algorithms.



%\item \href{http://www.haz.ca/}{Christian Muise}
% (\href{mailto:christian.muise@ibm.com}{christian.muise@ibm.com})\\
%Christian is a Research Staff Member at the AI Agent Design and Instantiation Lab of IBM
%in Cambridge, Massachusetts. He obtained his PhD from the University of Toronto under the
%supervision of Sheila McIlraith and J. Christopher Beck. His PhD research was on exploiting
%various notions of relevance in planning to synthesize and execute plans better. Following
%his PhD, he was a post-doc for two years with the University of Melbourne's Agentlab, and
%then subsequently a Research Fellow with the MERS group at MIT's CSAIL. He was co-publicity
%chair of ICAPS 2016, a co-organizer of the first IPC track on unsolvable planning problems,
%and has co-organized HSDIP in the past.


% \item \href{http://www.jordanthayer.com}{Jordan Thayer}
%   (\href{mailto:jthayer@sift.net}{jthayer@sift.net})\\ Jordan is
%   a researcher with Smart Information Flow Technologies (SIFT,
%   LLC). Before joining SIFT, he completed his Ph.D.  at the University
%   of New Hampshire under Wheeler Ruml.  His primary research
%   interests include heuristic search algorithms and formal
%   verification.  He has published several papers on heuristic search
%   applied to planning.  He was a co-organizer of the HSDIP workshop at
%   ICAPS 2012, 2013 and 2014.
%



% \item \href{http://www.hstairs.com/}{Enrico Scala}
% (\href{enricos83@gmail.com}{enricos83@gmail.com})\\
% Enrico is a Research Fellow at the Australian National University. He
% received his PhD in Computer Science in 2013 under the supervision of Pietro Torasso with a dissertation on robust plan execution for domains involving numeric resources. His current research is on optimal and satisficing hybrid planning via heuristic search and/or via compilation to Satifiability Modulo Theory.


% \item \href{https://fai.cs.uni-saarland.de/torralba/}{\'{A}lvaro
%   Torralba}
%   (\href{mailto:torralba@cs.uni-saarland.de}{torralba@cs.uni-saarland.de})\\ \'{A}lvaro
%   is a Research Fellow at Saarland University. He received his PhD
%   from Universidad Carlos III de Madrid in 2015 under the supervision
%   of Carlos Linares L\'{o}pez and Daniel Borrajo. His PhD dissertation
%   was on symbolic search and abstraction heuristics for cost-optimal
%   planning. His main research interests are in heuristic search and
%   automated planning.



%\item \href{http://ai.cs.unibas.ch/people/}{Silvan Sievers}
%  (\href{mailto:silvan.sievers@unibas.ch}{silvan.sievers@unibas.ch})\\
%Silvan is a Postdoctoral researcher at the University of Basel as part of the AI group
%led by Malte Helmert. His dissertation, which he recently defended at the
%University of Basel, deals with classical planning as heuristic search, with a
%focus on merge-and-shrink heuristics and symmetries. His main research
%interests are in heuristic search and automated planning.



\end{itemize}

\section*{Audience}
Previous years have shown an average number of about $11$ workshop
speakers and an audience of approximately $40$ attendees. The
expectation is for a similar amount this year.


\end{document}
