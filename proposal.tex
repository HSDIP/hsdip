\documentclass[10pt]{article}

\newcommand{\commentout}[1]{}
\usepackage[colorlinks=true,urlcolor=blue,citecolor=blue,pdfstartview=FitH]{hyperref}

\usepackage{geometry}
\usepackage{layout}
\usepackage{titling}

\setlength{\droptitle}{-5em}   % This is your set screw

\begin{document}

\title{ICAPS 2023 Workshop on Heuristics and Search for Domain-independent
Planning (HSDIP)%\\ \vspace*{0.7cm} Workshop Proposal
}
\date{}

\author{}

\maketitle

\vspace{-2cm}


We propose a continuation of the ICAPS workshop series ``Heuristics
and Search for Domain-independent Planning'' at ICAPS 2023.

\section*{Workshop Description}
Heuristic search is one of the main approaches in domain-independent
planning.  We look for contributions that help us better
understand the ideas underlying current heuristics, their limitations,
and ways for overcoming them.
%
Since the search algorithm plays an equally important role in the
approach, we also invite contributions on new ideas on search
techniques for domain-independent planning, as well as the synergy
between heuristics and search.

Contributions do not have to show that a new approach ``beats the competition''.
While performance measured in the number of evaluated nodes, time, and solution
quality remains relevant, in this workshop we seek above all crisp and
meaningful ideas and understanding.
%
We are interested in all variations of domain-independent planning
such as classical planning, temporal planning, hybrid planning, planning under
uncertainty, adversarial planning, or (model-based) reinforcement learning.

The HSDIP workshop has always been welcoming of multidisciplinary work,
for example, drawing inspiration from operations research (like row and
column generation algorithms), convex optimization (like gradient
optimization for hybrid planning), constraint programming or
satisfiability, or applications of machine learning in heuristic search
(e.g., learning heuristics, or heuristic selection). We will keep this
stance and promote it in the Call for Participation, particularly as ICAPS 2023
will continue the special track on planning \& learning.

\section*{Diversity}
The HSDIP organizing committee is pursuing several strategies to promote
diversity in the HSDIP community. So far, these include,

\begin{itemize}
  \item Inviting a diverse range of potential organizers for the workshop,
  both in terms of under-represented groups as well as academic progression.
  \item Committing to a diverse panel of speakers or invited talks (depending
  on the final chosen format).
  \item (if accessible) Actively promoting the workshop to interested undergraduate
  students specializing in AI, to better promote the field to a wider audience.
\end{itemize}

This list is by no means complete, and we will continue to innovate on what
we can do as an individual workshop to promote diversity both within the scope
of HSDIP, and ICAPS more broadly.

% The HSDIP workshop has always been and strives to continue being an inclusive environment.
% While a large portion of authors is still male and either European or North American/Australian,
% in the past we have seen a number of submissions from other, underrepresented regions,
% such as South America or Asia, or by non-male authors. We will continue encouraging
% diversity among submission authors.

% We feel that promoting diversity just by explicitly mentioning that we
% put an emphasis on underrepresented groups is not sufficient. We
% want to emphasize that a career choice as a computer scientist is a strong choice,
% regardless of gender and nationality. For that we plan to have an invited talk
% given by a speaker who represents that diversity and successful careers can go
% hand in hand.




\section*{Relevance to ICAPS 2023}

The workshop on heuristics and search for domain-independent planning (HSDIP) is
the successor of the workshop on heuristics for domain-independent planning
(HDIP), a biennial event that took place in conjunction with ICAPS during the
years 2007, 2009, and 2011. Starting with the HSDIP workshop in 2012, the
workshop series became an annual event.
Many ideas presented at HDIP and HSDIP workshops have led to contributions at major
conferences and pushed the frontier of research on heuristic planning in several
directions, both theoretically and practically. Throughout the years, HSDIP has
consistently had a large number of high-quality submissions and a large workshop
participation, including the virtual workshop from 2020 to 2022, which justifies its
continuation as an annual event.

\section*{Workshop Format}

The workshop is planned to have a full 1-day format, but the precise format is
to be decided as a function of the contributions received.  We aim to allow all
workshop contributors to present orally, and will adapt the presentation timing
to suit. In addition, we plan to include an invited talk and 1-2 dedicated
discussion sessions where the audience members are encouraged to participate.

We plan to organize the workshop in a small team to reduce the overhead. For
reviewing, we will take a major part of the involved effort. In order to handle
the workload and complement our expertise, we will invite 5-10 external
reviewers. These will be recruited mainly from past workshop organizers who are
familiar with the process for HSDIP. We expect to have around 10-20 submissions, which can
easily be handled in that way, particularly considering that reviewing for
workshops is generally lightweight.


\section*{Organizers}


\newcommand{\organizer}[4]{\href{#2}{#1} (\href{mailto:#3}{#3})\\{#4}}
\begin{itemize}

\item \organizer{Clemens B\"uchner}{https://ai.dmi.unibas.ch/people/buechner/}{clemens.buechner@unibas.ch}
Clemens B\"uchner is a PhD student at the University of Basel. His current
research focus lies in using ordered landmarks to guide heuristic search for
solving classical planning problems.

\item \organizer{Daniel Gnad}{https://rlplab.com/daniel-gnad/}{daniel.gnad@liu.de}
Daniel Gnad is a Postdoc at the Representation Learning and Planing Lab at
Link\"oping University. His research is in the field of classical AI planning,
mainly focusing on the newly developed concept of decoupled state-space search.
Besides, he has also worked on partial delete-relaxation and grounding
techniques.

\item \organizer{Thorsten Kl\"o\ss{}ner}{http://fai.cs.uni-saarland.de/kloessner/}{kloessner@cs.uni-saarland.de}
Thorsten Kl\"o\ss{}ner is a PhD student at the Foundations of Artificial
Intelligence group at Saarland University. His main research interest lies in
the construction of admissible heuristics, as well as admissible heuristic
combination strategies in Probabilistic Planning. In particular, his research
focuses primarily on abstraction heuristics, as well as cost partitioning
heuristics for stochastic shortest path problems.

\item \organizer{Sofia Lemons}{https://earlham.edu/faculty-staff/sofia-lemons/}{sofia.lemons@earlham.edu}
Sofia is an Assistant Professor in Computer Science at Earlham College. Her primary research focus in AI has been on heuristic search and creating fast systems for planning and decision-making. Specifically, her focus is on suboptimal and memory-limited heuristic search algorithms. She also specializes in mathematics and computer science education and pedagogy.


% \item \organizer{Salom\'{e} Eriksson}{https://ai.dmi.unibas.ch/people/eriksson/}{salome.eriksson@unibas.ch}
% Salom\'{e} Eriksson is a postdoctoral researcher at the University of Basel,
% where she also received her PhD in Computer Science in 2019. Her current
% research as well as her dissertation focuses on how we can create certificates
% for unsolvable planning tasks in order to increase trustworthiness of
% state-of-the-art planning systems.

% \item \organizer{Christian Muise}{http://www.haz.ca/}{christian.muise@queensu.ca}
% {Christian Muise is an assistant professor at Queen's University in
% Kingston, Canada. He completed his PhD on relevance-based techniques for
% automated planning at the University of Toronto. Subsequently, he
% held postdoc positions at the University of Melbourne and MIT,
% and then worked in industry for two years at the MIT-IBM Watson AI Lab. He
% has co-organized HSDIP in 2016, 2018 and has additionally held several
% ICAPS conference organization positions including Publicity Chair (2016, 2022),
% Journal Track Chair (2019), and Workshop Chair (2020).}

% \item \href{http://people.eng.unimelb.edu.au/nlipovetzky/}{Nir Lipovetzky}
%   (\href{mailto:nir.lipovetzky@unimelb.edu.au}{nir.lipovetzky@unimelb.edu.au})\\
% Nir is a Senior Lecturer at The University of Melbourne.
% His research is in the area of Automated Planning, with a special focus on width-based algorithms
% to introduce different approaches to the problem of inference in planning. He was co-publicity
% chair of ICAPS 2010, co-organizer of the HSDIP workshop at ICAPS 2015-2018, co-organizer of the
% first unsolvability IPC 2016 track, co-chair of ICAPS 2017 journal track, and Co-Program Chair ICAPS 2019.

%\item \href{http://ai.cs.unibas.ch/people/pommeren}{Florian Pommerening}
%   (\href{mailto:florian.pommerening@unibas.ch}{florian.pommerening@unibas.ch})\\
%Florian Pommerening is a postdoctoral researcher in the AI group at the
%University of Basel, Switzerland, where he completed his PhD in 2017.
%His main research interest is classical automated planning. For
%his PhD thesis, he used linear and mixed integer programs to
%automatically derive heuristic functions.

%\item \href{http://www.informatik.uni-freiburg.de/~speckd/}{David Speck}
%(\href{mailto:speckd@informatik.uni-freiburg.de}{speckd@informatik.uni-freiburg.de})\\
%David Speck is a PhD student in the Foundations of Artificial Intelligence
%group at the University of Freiburg, Germany. His current research focuses on
%the use of symbolic data structures for automated planning.

%\item \href{https://people.cs.aau.dk/~alto/}{{\'A}lvaro Torralba}
%  (\href{mailto:alto@cs.aau.dk}{alto@cs.aau.dk})\\ \'Alvaro Torralba is an Associate
%  Professor at Aalborg University. His main research interests are on heuristic search and
%  automated planning.


\end{itemize}

\section*{Estimated Interest}
Previous years have shown an average number of about $12$ workshop
speakers and an audience of approximately $40$ attendees. The
expectation is for a similar amount this year.


\end{document}
